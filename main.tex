\documentclass{article}
\usepackage{fullpage}
\usepackage[utf8]{inputenc}

\title{SOS Project Proposal\\
    \large{Neural Networks and Deep Learning}}
\author{Michelle Bergin}
\date{January 2018}

\begin{document}

\maketitle
\section{Overview}
My ultimate end goal is to have a deep learning machine demonstrate visually, emotions related to words and colors. In order to achieve this goal (whether it be during my time at Evergreen or later) I will need to research neural networks, and deep learning. I will need to have a database of the English language, and books. I will also need to create a survey where I can get people to associate feelings with words and colors. This info would also be fed to my machine. This is a BIG project. So I will be breaking it up into steps. My goal is not to do everything but to feel confident in what I have learned and accomplish some of these steps.

\section{Research}
I will need to read a LOT of tutorials, documents and research papers on neural networks and deep learning. This will take up a lot of my time. I will then make my own write up on what I have learned. I want to do this to instill my knew learned knowledge and to write it in such a way that is accessible to someone else who would like to learn as I did. This is a HUGE subject and I find it sincerely daunting. I want to help others feel more confident when they take their first steps into the chaos that is Deep Learning.

\section{Plug and Play Duplication}
When I feel confident with my knowledge and when I have a better idea of what I need to get started, I will find a neural network plug and play tutorial. In other words, I will find a getting started tutorial in which most to all the parts are provided that I can replicate. This is important so that I can actually get my feet dirty. I can actually see how a neural network works and in the future will be able to modify it to meet my needs. The reason I will be doing a neural network and not a deep learning tutorial is because there are very few deep learning tutorials that look like I could feel confident accomplishing. With TensorFlows website I feel confident I could create a deliverable.

\section{Data Collection: Books}
I need to get permissions to submit books to my neural network. More than likely I do not see this being an issue with the multitude of free public domain books.

\section{My Own Neural Network}
During this step I will feel confident enough with my knowledge of neural networks that I will create my own to process something simple like word associations based on locality. At this point I will have my database of English words, and I will have books that I will have my neural network process. In these books colors are used, I will create an algorithm to find words near colors and create associations. Before I move on from this step I want to be able to demonstrate visually (beautifully) the data procduced by this.

\section{Data Collection: Word Associations}
I need to create a survey where I get input from as many people as possible on their association of words, colors and feelings. Making the survey will be easy, surveymonkey and paper. Getting users is the hard part. I assume I will walk the halls of the school, ask friends, families, etc. Maybe I can even use one of Amazons sources for paying people pennies for filling out my surveys. I used to do them. I believe it's https://www.mturk.com/

\section{Deep Learning or More Neural Networks?}
Finally I want to implement more neural networks or deep learning in order to implement associations of feelings with words and colors from my data collection above. This may be done with only neural networks but it may require 'hidden' networks in a deep learning environment. At this point I may need to actually use AWS services or build my own machine. AWS would cost up to 100 dollars a month and building a deep learning machine can cost anywhere from 1000 - 7000 dollars. So this final step may be financially unfeasible.

\section{Time Line}
My goal is to get most of my research done from 4 - 8 weeks. This is a good time frame so that I have the last quarter to actually build a neural network. The sooner my research is done though I can get started on the tutorial plug and play. In that case I am hoping a month of research, a month of setting up and running a tensorflow example and then the rest of the time learning how to create my own and implementing my dictionary and word associations with books. 

\section{Links to Read for Research}
https://towardsdatascience.com/build-and-setup-your-own-deep-learning-server-from-scratch-e771dacaa252

https://www.analyticsvidhya.com/blog/2016/08/deep-learning-path/

http://www.deeplearningbook.org/

http://neuralnetworksanddeeplearning.com/

Microsoft: Deep Learning: Methods and Applications

http://deeplearning.net/tutorial/deeplearning.pdf

https://www.quora.com/Whats-the-most-effective-way-to-get-started-with-deep-learning

http://deeplearning.net/tutorial/gettingstarted.html

https://aws.amazon.com/blogs/machine-learning/get-started-with-deep-learning-using-the-aws-deep-learning-ami/

https://deeplearning4j.org/deeplearningforbeginners.html

https://www.lynda.com/Python-tutorials/Understand-deep-learning/550457/628324-4.html

https://www.lynda.com/Google-TensorFlow-tutorials/Building-Deploying-Applications-TensorFlow/601800-2.html

https://stats.stackexchange.com/questions/36247/how-to-get-started-with-neural-networks

https://medium.com/learning-new-stuff/how-to-learn-neural-networks-758b78f2736e

https://machinelearningmastery.com/tutorial-first-neural-network-python-keras/

https://www.tensorflow.org/tutorials/

\end{document}